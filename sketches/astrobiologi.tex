\documentclass[a4paper,11pt]{article}

\usepackage{revy}
\usepackage[utf8]{inputenc}
\usepackage[T1]{fontenc}
\usepackage[danish]{babel}

\revyname{Biorevy}
\revyyear{2016}
% HUSK AT OPDATERE VERSIONSNUMMER
\version{0.2}
\eta{$2$ minutter}
% \status{Ikke færdig}
\responsible{Thomas BT}

\title{Astrobiologi 2.0}
\author{Helene, Dannen, Stine}

\begin{document}
\maketitle

\begin{roles}
	\role{A}[Veronica] Astrolog
	\role{B}[Danny] Biolog
	\role{Ninja}[Jeppe] Sceneninja
\end{roles}

\begin{props}
	\prop{2 stole}[skaffer selv]
	\prop{1 bord}[Person der skaffer]
\end{props}


\begin{sketch}

\scene{Arbejdsgiver sidder (evt. ved et bord), klar til jobsamtale}
\scene{lys op}
\scene{B kommer ind, giver hånd, sætter sig ned.}
\says{A} Nå, men lad os så høre: hvad gør dig kvalificeret til at arbejde her? 
\says{B} Som du nok har læst, så siger mit CV Biologi + Fysik = superkompetent. \act{peger på sig selv} 
\says{A} Okay, men vi går ikke så nøje op i universitetsgrader 
\says{B} Det var godt, for jeg fik aldrig helt min kandidatgrad.
\says{A} Der er så mange andre ting der er vigtige -- en videnskabelig tilgang til astronomiske forhold og en lyst til at gøre en forskel. Nå, men jeg kan se i dit CV at du har arbejdet meget med løver? 
\says{B} Ja, jeg startede på at skrive et speciale om løver, med feltarbejde i Afrika. Løver + Afrika = pisse fedt. Jeg studerede især deres kønsroller, og gruppestrukturerne. 
\says{A} Ja, det er jo meget relevant. Men hvordan ville du beskrive løvers personlighed? 
\says{B} Øhh, jo, tjoeh. Mange ser jo løver som majestætiske, men faktisk sover de op til 20 timer i døgnet. Så jeg vil nok mere kalde dem dovne. Derudover er de meget patriarkalske, men det kvinderne der gør alt arbejdet. Og så er de jo nogle glubske rovdyr. 
\says{A} Utraditionelt syn på dem, men ja, jo… Vi arbejder jo også meget med himmellegemer. Hvad kan du fortælle om betydningen af stjerner og planeters struktur. 
\says{B} Ja, altså, stjerners grundstofsammensætning har jo betydning for farven af . . . 
\says{A} Uuh, ja, det var ikke helt sådan jeg tænkte. Jeg mente mere, hvad er BETYDNINGEN af stjerner og planeters struktur i forhold til løvers personlighed?
\says{B} Ja, der ved jeg jo lidt -- stjerner + planeter + løver = ????. Men hvor er det så jeg som adfærdsbiolog kommer ind?
\says{A} Ja, men det er nemt nok: Forskelle på stjerner og planeter har jo indflydelse på løvers personlighed. 
\says{B} Men, hvordan ved I det? Skal vi ikke finde løver i rummet før vi kan kigge på deres adfærd?
\says{A} Nej, nej, det behøver man ikke
\says{B} Men I er jo astrobiologer?
\says{A} Jo. Biologi + astrologi = Stjernetegn. 
\scene{lys ned}

\end{sketch}
\end{document}
